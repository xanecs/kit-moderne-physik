%% Generated by Sphinx.
\def\sphinxdocclass{report}
\documentclass[a4paper,10pt,ngerman]{sphinxmanual}
\ifdefined\pdfpxdimen
   \let\sphinxpxdimen\pdfpxdimen\else\newdimen\sphinxpxdimen
\fi \sphinxpxdimen=.75bp\relax

\PassOptionsToPackage{warn}{textcomp}
\usepackage[utf8]{inputenc}
\ifdefined\DeclareUnicodeCharacter
 \ifdefined\DeclareUnicodeCharacterAsOptional
  \DeclareUnicodeCharacter{"00A0}{\nobreakspace}
  \DeclareUnicodeCharacter{"2500}{\sphinxunichar{2500}}
  \DeclareUnicodeCharacter{"2502}{\sphinxunichar{2502}}
  \DeclareUnicodeCharacter{"2514}{\sphinxunichar{2514}}
  \DeclareUnicodeCharacter{"251C}{\sphinxunichar{251C}}
  \DeclareUnicodeCharacter{"2572}{\textbackslash}
 \else
  \DeclareUnicodeCharacter{00A0}{\nobreakspace}
  \DeclareUnicodeCharacter{2500}{\sphinxunichar{2500}}
  \DeclareUnicodeCharacter{2502}{\sphinxunichar{2502}}
  \DeclareUnicodeCharacter{2514}{\sphinxunichar{2514}}
  \DeclareUnicodeCharacter{251C}{\sphinxunichar{251C}}
  \DeclareUnicodeCharacter{2572}{\textbackslash}
 \fi
\fi
\usepackage{cmap}
\usepackage[T1]{fontenc}
\usepackage{amsmath,amssymb,amstext}
\usepackage{babel}
\usepackage{times}
\usepackage[Sonny]{fncychap}
\usepackage{sphinx}

\usepackage{geometry}

% Include hyperref last.
\usepackage{hyperref}
% Fix anchor placement for figures with captions.
\usepackage{hypcap}% it must be loaded after hyperref.
% Set up styles of URL: it should be placed after hyperref.
\urlstyle{same}

\addto\captionsngerman{\renewcommand{\figurename}{Abb.}}
\addto\captionsngerman{\renewcommand{\tablename}{Tab.}}
\addto\captionsngerman{\renewcommand{\literalblockname}{Quellcode}}

\addto\captionsngerman{\renewcommand{\literalblockcontinuedname}{Fortsetzung der vorherigen Seite}}
\addto\captionsngerman{\renewcommand{\literalblockcontinuesname}{Fortsetzung auf der nächsten Seite}}

\addto\extrasngerman{\def\pageautorefname{Seite}}

\setcounter{tocdepth}{1}



\title{Moderne Physik für Informatiker Documentation}
\date{16.04.2018}
\release{Sommersemester 2018}
\author{Leon Bentrup}
\newcommand{\sphinxlogo}{\vbox{}}
\renewcommand{\releasename}{Release}
\makeindex

\begin{document}
\ifnum\catcode`\"=\active\shorthandoff{"}\fi
\maketitle
\sphinxtableofcontents
\phantomsection\label{\detokenize{index::doc}}



\chapter{Inhalt}
\label{\detokenize{index:moderne-physik-fur-informatiker}}\label{\detokenize{index:inhalt}}

\section{Klassische Mechanik}
\label{\detokenize{klassischemechanik:klassische-mechanik}}\label{\detokenize{klassischemechanik::doc}}

\subsection{Abriss der Newtonschen Mechanik}
\label{\detokenize{klassischemechanik:abriss-der-newtonschen-mechanik}}
\sphinxstylestrong{Problemstellung} der Mechanik:

Orte \(\vec{r_i}\) und Geschwindigkeiten \(\vec{v_i}\)
zur Zeit \(t_0\) gegeben für ein System von Massepunkten,
\(1 \le i \le N\)

Es wirken Kräfte \(\vec{F_j}\) auf die Massepunkte und ggf.
Kräfte \(\vec{F_{ij}}\) zwischen den Massepunkten.

Wie lauten die \sphinxstylestrong{kinematischen Größen} \(\vec{r_i}(t)\)
und \(\vec{v_i}(t) = \dot{\vec{r_i}}(t)\) für beliebige
Zeiten \(t\) danach?

Die kinematischen Größen \(\vec{r_i}(t)\) und \((\dot{\vec{r_i}}, \ddot{\vec{r_i}}, ...)\)
werden als Lösungen gewöhnlicher Differentialgleichungen gefunden,
das sind die \sphinxstylestrong{Bewegungsgleichungen}.

\begin{sphinxShadowBox}
\sphinxstylesidebartitle{Notation}
\begin{align*}\!\begin{aligned}
\dot{\vec{r_i}} = \frac{d}{dt}\vec{r_i}\\
(\frac{dx_i}{dt}, \frac{dy_i}{dt}, \frac{dz_i}{dt})\\
\end{aligned}\end{align*}\end{sphinxShadowBox}

Neben den Kinemstischen Größen gibt es die wichtigen Begriffe
\sphinxstylestrong{Kraft}, \sphinxstylestrong{Masse}, \sphinxstylestrong{Impuls}

Kraft: Vektorielle Größe \(\vec{F}\)

Ursache der Bewegung oder: Änderung des Bewegungszustandes.

\(\leadsto\) Kräftefrei \(\rightarrow\) Bewegung unverändert.

\(\rightarrow\) \sphinxstylestrong{Newtonsche Gesetze}


\subsubsection{lex prima: Galileisches Trägheitsgesetz}
\label{\detokenize{klassischemechanik:lex-prima-galileisches-tragheitsgesetz}}
Es gibt \sphinxstylestrong{Inertialsysteme}, in denen ein kräftefreier Körper
(= Massepunkt) ruht, oder sich gradlinig, gleichförmig bewegt.

\sphinxstylestrong{Definition:} Jeder Massepunkt setzt der Einwirkung von
Kräften einen Trägheitswiderstand entgegen. = träge Masse (:math:m)

Damit Impuls \(\vec{p}=m\vec{v}\)


\subsubsection{lex secunda: Bewegungsgesetz}
\label{\detokenize{klassischemechanik:lex-secunda-bewegungsgesetz}}\begin{equation*}
\begin{split}\dot{\vec{p}} = \vec{F}\end{split}
\end{equation*}
Meist \(m\) unveränderlich
\begin{align*}\!\begin{aligned}
\dot{\vec{v}} = \frac{d}{dt}\vec{v} &= \vec{a} & \mathrm{(Beschleunigung)}\\
m\vec{a} &= \vec{F}\\
\end{aligned}\end{align*}

\subsubsection{lex tertie: Actio = Reactio}
\label{\detokenize{klassischemechanik:lex-tertie-actio-reactio}}\begin{equation*}
\begin{split}\vec{F_{ij}} = - \vec{F_{ji}}\end{split}
\end{equation*}
\(\rightarrow\) Festlegung der trägen Masse unabhängig von
der Kraft

{\color{red}\bfseries{}:todo:{}`Grafik{}`}

Spannen, loslassen \(\Rightarrow \vec{v_1},\vec{v_2}\)

\(\Rightarrow\) Verhältnis der Geschwindigkeiten
\begin{align*}\!\begin{aligned}
{\lvert}\vec{v_i}{\rvert} &= v_i\\
\frac{v_1}{v_2} &= k(\frac{m_1}{m_2})\\
\end{aligned}\end{align*}
hängt nicht von Kraft oder Feder ab.

\(\rightarrow\) Definition Masse als \sphinxstylestrong{Basiseinheit}



\renewcommand{\indexname}{Stichwortverzeichnis}
\printindex
\end{document}